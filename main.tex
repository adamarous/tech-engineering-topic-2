\documentclass[12pt]{article}

\title{Tech and engineering}
\author{Adam Martinez}
\date{\today}

\begin{document}

\maketitle

\section*{Topic 2}

\subsection*{Tech and sustainability}

\subsubsection*{Sustainable development}

Effectively use resources available today without compromising the future.

\subsubsection*{Sustainable technology}

They minimise the environmental impact of activities through reusing, recycling,
and reducing the use of resources and energy.
They contribute towards SDGs.

\subsubsection*{Technical contributions}

\begin{itemize}
    \item Energetic efficency
    \begin{itemize}
        \item Sensors can reduce energy consumption.
    \end{itemize}
    \item Renewable energy
    \begin{itemize}
        \item They generate clean, reliable energy.
    \end{itemize}
    \item Environmental control
    \begin{itemize}
        \item Sensors allow quality checking, and thus improvements.
    \end{itemize}
    \item Circular economy
    \begin{itemize}
        \item Maximise the reincorporation of materials into the production
        chain.
    \end{itemize}
    \item Sustainable product development
    \begin{itemize}
        \item E.g. nanotech creates more environmental-friednly products.
    \end{itemize}
\end{itemize}

\subsubsection*{Tech trends towards the environment}

These are only applicable if policies are implemented to regulate them. They are
also very dependent on the population's awareness.

\begin{itemize}
    \item Reneweable energies
    \begin{itemize}
        \item They reduce greenhouse gases; research is being done into their
        distribution and storage.
    \end{itemize}
    \item IoT
    \begin{itemize}
        \item They optimise the use of resources by means of a more intelligent
        net of control.
    \end{itemize}
    \item Blockchain
    \begin{itemize}
        \item They authenticate immutable registres for processes and products.
    \end{itemize}
    \item AI
    \begin{itemize}
        \item They provide resource optimisation and pattern identification for
        easier decision-making.
    \end{itemize}
    \item Circular economy
    \begin{itemize}
        \item The idea of reincorporation is being implemented through e.g. 3D
        printing.
    \end{itemize}
    \item Sustainable mobility
    \begin{itemize}
        \item Electric vehicles, public transport, and shared mobility reduce
        the environmental impact of transportation.
    \end{itemize}
\end{itemize}

\subsection*{Sustainable materials}

\begin{itemize}
    \item They don't deplete natural resources.
    \item Their emissions are lower.
    \item Their waste is lower.
    \item They can be reused or recycled.
\end{itemize}

\subsubsection*{Types of sustainable materials}

\begin{itemize}
    \item Recyclable, they can be given a different use.
    \item Biodegradable, they can be decomposed by natural processes.
    \item Reusable, they can be used again for the same purpose.
    \item Vegetal, they come from renewable sources that spend less energy and
    leave less waste.
\end{itemize}

\subsubsection*{Pros of sustainable materials}

\begin{itemize}
    \item Water and air are better preserved, because toxins are released less
    often.
    \item Energy savings are higher.
    \item They preserve the environment because they don't deplete natural
    resources.
    \item They reduce waste and allow for recycling and reuse.
\end{itemize}

\subsection*{Woods}



\end{document}