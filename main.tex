\documentclass{article}

\title{Tech and engineering}
\author{Adam Martinez}
\date{}

\begin{document}

\maketitle

\section*{Topic 2}

\subsection*{Tech and sustainability}

\subsubsection*{Sustainable development}

Effectively use resources available today without compromising the future.

\subsubsection*{Sustainable technology}

They minimise the environmental impact of activities through reusing, recycling,
and reducing the use of resources and energy.
They contribute towards SDGs.

\subsubsection*{Technical contributions}

\begin{itemize}
    \item Energetic efficency
    \begin{itemize}
        \item Sensors can reduce energy consumption.
    \end{itemize}
    \item Renewable energy
    \begin{itemize}
        \item They generate clean, reliable energy.
    \end{itemize}
    \item Environmental control
    \begin{itemize}
        \item Sensors allow quality checking, and thus improvements.
    \end{itemize}
    \item Circular economy
    \begin{itemize}
        \item Maximise the reincorporation of materials into the production
        chain.
    \end{itemize}
    \item Sustainable product development
    \begin{itemize}
        \item E.g. nanotech creates more environmental-friednly products.
    \end{itemize}
\end{itemize}

\subsubsection*{Tech trends towards the environment}

These are only applicable if policies are implemented to regulate them. They are
also very dependent on the population's awareness.

\begin{itemize}
    \item Reneweable energies
    \begin{itemize}
        \item They reduce greenhouse gases; research is being done into their
        distribution and storage.
    \end{itemize}
    \item IoT
    \begin{itemize}
        \item They optimise the use of resources by means of a more intelligent
        net of control.
    \end{itemize}
    \item Blockchain
    \begin{itemize}
        \item They authenticate immutable registres for processes and products.
    \end{itemize}
    \item AI
    \begin{itemize}
        \item They provide resource optimisation and pattern identification for
        easier decision-making.
    \end{itemize}
    \item Circular economy
    \begin{itemize}
        \item The idea of reincorporation is being implemented through e.g. 3D
        printing.
    \end{itemize}
    \item Sustainable mobility
    \begin{itemize}
        \item Electric vehicles, public transport, and shared mobility reduce
        the environmental impact of transportation.
    \end{itemize}
\end{itemize}

\subsection*{Sustainable materials}

\begin{itemize}
    \item They don't deplete natural resources.
    \item Their emissions are lower.
    \item Their waste is lower.
    \item They can be reused or recycled.
\end{itemize}

\subsubsection*{Types of sustainable materials}

\begin{itemize}
    \item Recyclable, they can be given a different use.
    \item Biodegradable, they can be decomposed by natural processes.
    \item Reusable, they can be used again for the same purpose.
    \item Vegetal, they come from renewable sources that spend less energy and
    leave less waste.
\end{itemize}

\subsubsection*{Pros of sustainable materials}

\begin{itemize}
    \item Water and air are better preserved, because toxins are released less
    often.
    \item Energy savings are higher.
    \item They preserve the environment because they don't deplete natural
    resources.
    \item They reduce waste and allow for recycling and reuse.
\end{itemize}

\subsection*{Woods}

They are generated from the leftovers in the sawmill.

\subsubsection*{Types of woods}

\begin{itemize}
    \item Laminated, they are made from thin layers of grain.
    \item Chipboard, they are made from crushed chips.
    \item Fibreboard, they are made from pressed fibres.
    \item Striped board, they are made from glued pieces of the same wood.
\end{itemize}

\subsection*{Properties of materials}

\begin{itemize}
    \item Sensory properties, they are perceived by the senses.
    \item Thermal properties, they are the materials' response to heat.
    \item Magnetic properties, they are the materials' capacity to be attracted
    by a magnet.
    \item Technological properties, they are the materials' response to
    production processes.
    \begin{itemize}
        \item Fusibility, they are the materials' capacity to be melted.
        \item Ductility, they are the materials' capacity to be stretched in
        threads.
        \item Malleability, they are the materials' capacity to be shaped into
        thin layers.
        \item Plasticity, they are the materials' capacity to be shaped without
        breaking.
    \end{itemize}
    \item Ecological properties, they are the materials' harmfulness to the
    environment.
\end{itemize}

\subsubsection*{Chemical properties}

\begin{itemize}
    \item Chemical stability, the compound's need for an external agent to
    react or the result of the reaction with another compound.
    \item Oxidation, the reaction with oxygen accelerated by heat provides
    protection against corrosion.
    \item Corrosion, oxidation in a humid environment or with aggresive
    substances.
\end{itemize}

\subsubsection*{Physical properties}

\begin{itemize}
    \item Density, the mass per unit of volume.
    \item Electrical resistance, the opposition to the passage of electric
    current measured in ohms, resistivity and conductivity; the materials can be
    conductors, semiconductors or insulators.
    \item Optical properties, the materials' response to light; they can be
    transparent, translucent, or opaque.
\end{itemize}

\subsubsection*{Mechanical properties}

\begin{itemize}
    \item Hardness, the materials' resistance to being scratched or cut.
    \item Tenacity, the materials' resistance to breaking when hit; the lesser
    ones are fragile.
    \item Flexibility, the materials' capacity to be bent without breaking; the
    lesser ones are rigid.
    \item Elasticity, the materials' capacity to return to their original shape
    after being deformed; the lesser ones are plastic.
\end{itemize}

\subsection*{Destructive tests}

\subsubsection*{Tensile test}

It uses a stress-strain diagram.

\begin{itemize}
    \item Tensile force \emph{F} (in N, kgf, or kp).
    \item Elongation \\ $\Delta L = L - L_0$ (in mm) \\ $\epsilon =
    \frac{\Delta L}{L_0}$ (non-dimensional).
    \item Strain \\ $\sigma = \frac{F}{S_0}$ (in Pa, N/mm$^2$, kp/cm$^2$) \\
    $\sigma = E$ (modulus of elasticity) $\cdot \: \epsilon$ (only in the
    proportional area).
    \item Max elongation $A\left(\%\right) = 100
    \left(\frac{L_f - L_0}{L_0}\right)$.
    \item Area of a cylindre $\pi r^2 = \pi \frac{d}{2}^2$.
    \item Area under the line...
    \begin{itemize}
        \item Under the tensile test $\rightarrow$ Tenacity.
        \item Under the elastic limit $\rightarrow$ Resillience.
        \item Under the force-elongation curve $\rightarrow$ Work.
        \item Ductility \% relative elongation $\rightarrow$ Plastic deformation
        until fracture.
        \item Rigidness $\rightarrow$ Plastic deformation until fracture;
        proportional to \emph{E}.
    \end{itemize}
    \item Maximum tensile force $\rightarrow$ Maximum charge limit on a material
    smaller than the tensile tension on the proportionality limit.
    \begin{itemize}
        \item The material doesn't suffer plastic deformation.
        \item The material abides by Hook's law.
        \item The process's security is more asured.
    \end{itemize}
    \item Coefficient of secutiry $\rightarrow$ $n = \frac{\sigma_f}{\sigma_w}$,
    where $\sigma_f$ is the creep stress and $\sigma_w$ is the work stress.
\end{itemize}

\subsubsection*{Hardness test}

\begin{itemize}
    \item Brinnel
    \begin{itemize}
        \item Not too hard materials.
        \item Tempered steel ball.
        \item Measures the diameter of the indentation on mid-size materials.
        \item \emph{HB} $= \frac{F}{S} = \frac{2F}{\pi D\left(D - \sqrt{D^2 d^2}
        \right) }$ (in $\frac{kp}{mm^2}$).
        \item Expressed in \emph{HB}, \emph{mm}, \emph{kp}, \emph{s}.
    \end{itemize}
    \item Rockwell
    \begin{itemize}
        \item Hard and soft materials.
        \item Diamond cone or steel ball.
        \item Measures the depth of the indentation and is the quickest.
        \item Soft materials $\rightarrow$ \emph{HR} $= 100 - e$ \\ Hard
        materials $\rightarrow$ \emph{HR} $= 130 - e$ \\ where \emph{e} is the
        permanent indentation.
    \end{itemize}
    \item Vickers
    \begin{itemize}
        \item Very hard materials.
        \item Diamond pyramid.
        \item Measures the diameter of the indentation and is very expensive.
        \item \emph{HV} $= \frac{1.854 F}{d^2}$ (in $\frac{kp}{mm^2}$).
    \end{itemize}
\end{itemize}

\subsubsection*{Resillience test}

\begin{itemize}
    \item Charpy test.
    \item Test piece $10 \times 10$ mm, $L = 55$ mm with 2 mm U/V serration
    ($S_0 = 80$ mm$^2$).
    \item $\rho = \frac{m g \left(h_0 - h_f\right)}{S}$ (in $\frac{J}{m^2}$).
\end{itemize}

\subsubsection*{Fatigue test}

The material is subjected to cycli loads without reaching the breaking point.
The most common one is the rotative bending test.
Steel and other materials have a fatigue limit around 0.4 - 0.5.

\subsection*{Non-destructive tests}

\subsubsection*{Ultrasonic test}

\begin{itemize}
    \item Uses high-frequency sound waves.
    \item A transducer applies sound waves into the steel. The sound waves
    travel through the material and reflect back if there's a discontinuity.
    \item Detects internal flaws, measures thickness, and finds changes in
    material properties.
\end{itemize}

\subsubsection*{Radiographic test}

\begin{itemize}
    \item Uses X-rays or gamma rays.
    \item Radiographic films are exposed to a radiation source passed through
    the steel. The film shows the internal structure based on varying radiation
    absorption levels.
    \item Identifies internal defects like cracks, voids, and inclusions.
\end{itemize}

\subsubsection*{Magnetic particle test}

\begin{itemize}
    \item Magnetic induction.
    \item The steel is magnetized. Fine magnetic particles are applied to the
    surface, which gather at discontinuities, visible under proper lighting
    conditions.
    \item Detects surface and slightly subsurface discontinuities in
    ferromagnetic materials.
\end{itemize}

\subsubsection*{Eddy Current test}

\begin{itemize}
    \item Electromagnetic induction.
    \item Alternating current is passed through a coil, creating an alternating
    magnetic field that induces eddy currents in the conductive steel.
    Variations in the eddy current flow are monitored to detect flaws.
    \item Detects surface and near-surface defects, measures coating
    thicknesses, and conducts conductivity measurements.
\end{itemize}

\end{document}